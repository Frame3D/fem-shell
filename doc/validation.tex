\section{Validation}
In the scope of this thesis a program was developed that implements flat shell elements. Two different discretizations of shell elements are provided: A three-node triangular element and a four-node quadrilateral element. A plane and a plate bending element were superimposed to the final flat shell element. This chapter attends the validation, i.e. the accuracy and convergence properties, of the two shell element discretizations as well as their parts, the plane and the plate element. Several problems were chosen that show the correctness of the discretized elements. The tests were taken from different sources, namely \cite{kansara2004development}, \cite{macneal1985proposed}, \cite{wilson1996three} and \cite{jin1994analysis}. Many example problems have analytical results to be compared with, the examples from \cite{kansara2004development} used a commercial software called \textit{SAP2000} \cite{sap-2000} as a benchmark for comparison. Since this software is used in practice for over 30 years, it will be used here as a benchmark as well.
 \subsection{Test A: Plane Displacement with Tri-3 Element}
  The three-node triangular plane element \textbf{Tri-3} is validated with a cantilever beam shown in Figure \ref{fig:testA}. The example problem was taken from \cite{kansara2004development} (Test Example 2).
 \begin{figure}[htbp]
   	\centering
	\setlength\unitlength{1.65cm}
   	\begin{picture}(9,3)
   	\thicklines
   	\put(4.51,1.51){\vector(1,0){0.75}}
   	\put(4.52,1.5){\vector(0,1){0.75}}
   	\put(5.07,1.27){$\mathbf{x}$}
   	\put(4.55,2.13){$\mathbf{y}$}   	
   	\put(8.52,0.0){\vector(0,1){0.5}}
   	\put(8.52,1.0){\vector(0,1){0.5}}
   	\put(8.52,2.0){\vector(0,1){0.5}}   	
   	\thinlines
   	\polygon(0.5,0.5)(0.5,2.5)(8.5,2.5)(8.5,0.5)
   	\polyline(0.5,1.5)(1.5,2.5)(1.5,0.5)(0.5,1.5)(1.5,1.5)
   	\polyline(1.5,1.5)(2.5,2.5)(2.5,0.5)(1.5,1.5)(2.5,1.5)
   	\polyline(2.5,1.5)(3.5,2.5)(3.5,0.5)(2.5,1.5)(3.5,1.5)
   	\polyline(3.5,1.5)(4.5,2.5)(4.5,0.5)(3.5,1.5)(4.5,1.5)
   	\polyline(4.5,1.5)(5.5,2.5)(5.5,0.5)(4.5,1.5)(5.5,1.5)
   	\polyline(5.5,1.5)(6.5,2.5)(6.5,0.5)(5.5,1.5)(6.5,1.5)
   	\polyline(6.5,1.5)(7.5,2.5)(7.5,0.5)(6.5,1.5)(7.5,1.5)
   	\polyline(7.5,1.5)(8.5,2.5)(8.5,0.5)(7.5,1.5)(8.5,1.5)   	
   	\Line(0.5,0.5)(0.3,0.7) \Line(0.5,1)(0.3,1.2) \Line(0.5,1.5)(0.3,1.7) \Line(0.5,2)(0.3,2.2) \Line(0.5,2.5)(0.3,2.7)   	
   	\put(8.6,0.15){$6\frac{2}{3}$}
   	\put(8.6,1.15){$26\frac{2}{3}$}
   	\put(8.6,2.15){$6\frac{2}{3}$}   	
   	\put(0.56,0.55){$0$} \put(1.56,0.55){$1$} \put(2.56,0.55){$2$} \put(3.56,0.55){$3$} \put(4.56,0.55){$4$} \put(5.56,0.55){$5$} \put(6.56,0.55){$6$} \put(7.56,0.55){$7$} \put(8.56,0.55){$8$}
   	\put(0.56,1.55){$9$}  \put(1.56,1.55){$10$} \put(2.56,1.55){$11$} \put(3.56,1.55){$12$} \put(4.56,1.55){$13$} \put(5.56,1.55){$14$} \put(6.56,1.55){$15$} \put(7.56,1.55){$16$} \put(8.56,1.55){$17$}
   	\put(0.56,2.55){$18$} \put(1.56,2.55){$19$} \put(2.56,2.55){$20$} \put(3.56,2.55){$21$} \put(4.56,2.55){$22$} \put(5.56,2.55){$23$} \put(6.56,2.55){$24$} \put(7.56,2.55){$25$} \put(8.56,2.55){$26$}
   	\end{picture}
   	\caption{Cantilever beam consisting of 32 triangular elements, clamped at the left side and a total force of 40 kips applied in positive y-direction at the right side}
   	\label{fig:testA}
   \end{figure}
      
   \begin{itemize}
   \item \textbf{Mesh dimensions}\\
   Length $l = 48\ inch$\\
   Depth $h = 12\ inch$\\
   Thickness $t = 1\ inch$
   
   \item \textbf{Material properties}\\
   Young's Modulus $E = 30000 ksi$\\
   Poisson's ratio $\nu = 0.25$
   
   \item \textbf{Boundary conditions}\\
   Clamped boundary conditions at node 0, 9 and 18, i.e.\ left side of the cantilever beam.
   
   \item \textbf{Loading}\\
   A concentrated load of $40 kips$ in total. Node 8 and 26 has a load of $6 \frac{2}{3}$, node 17 has a load of $26 \frac{2}{3}$.
   \end{itemize}
   
   \paragraph{Results:} The displacements in x- and y-direction at node 22 and 26 are presented in Table \ref{tab:testA} together with the results from the \textit{SAP2000} software presented in \cite{kansara2004development}. The displacements of the thesis' program deviate from the commercial software in all cases for at most $0.027\%$. The triangle orientation in the example mesh contains both, the square diagonal facing the upper right and facing the lower right corner of the square. To show that the mixed usage of these orientation types increases the accuracy, two additional tests were made, one with only triangles having their hypotenuse facing the upper right corner of the square ($\boxslash$) and one with only triangles of the other type ($\boxbackslash$). The results of those tests can only be compared to the first test's results of the program, since no benchmark values are available. What can first be seen is that either of the new variants is less accurate than the mixed version. Second, the $\boxslash$-variant is more accurate than the $\boxbackslash$-variant and third, the accuracy in x-direction is better than in y-direction, particularly for the $\boxbackslash$-orientation.
   
   \begin{table}[htbp]
   \centering
   \begin{tabular}{c|c|C{2.5cm}|C{2.5cm}|c}
   \textbf{Node} & \textbf{Displacement} & \textbf{Results from program$^{(a)}$} & \textbf{Results from SAP2000$^{(b)}$} & \textbf{Difference (\%)}\\\hline\hline
   \multirow{2}{*}{22} & $u_x$ & $-0.0255988$ & $-0.025605$ & $0.024\%^{(b)}$\\
                       & $u_y$ & $ 0.0629549$ & $ 0.062971$ & $0.026\%^{(b)}$\\\hline
   \multirow{2}{*}{26} & $u_x$ & $-0.0342621$ & $-0.034271$ & $0.027\%^{(b)}$\\
                       & $u_y$ & $ 0.1944070$ & $ 0.194456$ & $0.025\%^{(b)}$\\\hline\hline
   \multirow{2}{*}{$\boxslash$ 22}     & $u_x$ & $-0.0243863$ & - & $4.97\%^{(a)}$\\
                                         & $u_y$ & $ 0.0552195$ & - & $14.01\%^{(a)}$\\\hline
   \multirow{2}{*}{$\boxslash$ 26}     & $u_x$ & $-0.0328891$ & - & $4.17\%^{(a)}$\\
                                         & $u_y$ & $ 0.1829420$ & - & $6.27\%^{(a)}$\\\hline\hline
   \multirow{2}{*}{$\boxbackslash$ 22} & $u_x$ & $-0.0235617$ & - & $8.65\%^{(a)}$\\
                                         & $u_y$ & $ 0.0440028$ & - & $43.07\%^{(a)}$\\\hline
   \multirow{2}{*}{$\boxbackslash$ 26} & $u_x$ & $-0.0322955$ & - & $6.09\%^{(a)}$\\
                                         & $u_y$ & $ 0.1564130$ & - & $24.29\%^{(a)}$\\\hline
   \end{tabular}
   \caption{Displacements and deviations for Test A}
   \label{tab:testA}
   \end{table}
   
   
 \subsection{Test B: Plane Displacement with Quad-4}
  The four-node quadrilateral plane element \textbf{Quad-4} is validated with the same cantilever beam that was used in Test A. It is shown in Figure \ref{fig:testB}. The example problem was also taken from \cite{kansara2004development} (Test Example 5).
  \begin{figure}[htbp]
    \centering
  	\setlength\unitlength{1.65cm}
   	\begin{picture}(9,3)
   	\thicklines
   	\put(4.51,1.51){\vector(1,0){0.75}}
   	\put(4.52,1.5){\vector(0,1){0.75}}
   	\put(5.07,1.27){$\mathbf{x}$}
   	\put(4.55,2.13){$\mathbf{y}$}   	
   	\put(8.52,0.0){\vector(0,1){0.5}}
   	\put(8.52,1.0){\vector(0,1){0.5}}
   	\put(8.52,2.0){\vector(0,1){0.5}}   	
   	\thinlines
   	\polygon(0.5,0.5)(0.5,2.5)(8.5,2.5)(8.5,0.5)
   	\Line(0.5,1.5)(8.5,1.5)
   	\Line(1.5,0.5)(1.5,2.5) \Line(2.5,0.5)(2.5,2.5) \Line(3.5,0.5)(3.5,2.5) \Line(4.5,0.5)(4.5,2.5) \Line(5.5,0.5)(5.5,2.5) \Line(6.5,0.5)(6.5,2.5) \Line(7.5,0.5)(7.5,2.5)
   	\Line(0.5,0.5)(0.3,0.7) \Line(0.5,1)(0.3,1.2) \Line(0.5,1.5)(0.3,1.7) \Line(0.5,2)(0.3,2.2) \Line(0.5,2.5)(0.3,2.7)   	
   	\put(8.6,0.15){$6\frac{2}{3}$}
   	\put(8.6,1.15){$26\frac{2}{3}$}
   	\put(8.6,2.15){$6\frac{2}{3}$}   	
   	\put(0.56,0.55){$0$} \put(1.56,0.55){$1$} \put(2.56,0.55){$2$} \put(3.56,0.55){$3$} \put(4.56,0.55){$4$} \put(5.56,0.55){$5$} \put(6.56,0.55){$6$} \put(7.56,0.55){$7$} \put(8.56,0.55){$8$}
   	\put(0.56,1.55){$9$}  \put(1.56,1.55){$10$} \put(2.56,1.55){$11$} \put(3.56,1.55){$12$} \put(4.56,1.55){$13$} \put(5.56,1.55){$14$} \put(6.56,1.55){$15$} \put(7.56,1.55){$16$} \put(8.56,1.55){$17$}
   	\put(0.56,2.55){$18$} \put(1.56,2.55){$19$} \put(2.56,2.55){$20$} \put(3.56,2.55){$21$} \put(4.56,2.55){$22$} \put(5.56,2.55){$23$} \put(6.56,2.55){$24$} \put(7.56,2.55){$25$} \put(8.56,2.55){$26$}
   	\end{picture}
   	\caption{Cantilever beam consisting of 16 quadrilateral elements, clamped at the left side and a total force of 40 kips applied in positive y-direction at the right side}
   	\label{fig:testB}
  \end{figure}

  \begin{itemize}
   \item \textbf{Mesh dimensions}\\
   Length $l = 48\ inch$\\
   Depth $h = 12\ inch$\\
   Thickness $t = 1\ inch$

   \item \textbf{Material properties}\\
   Young's Modulus $E = 30000 ksi$\\
   Poisson's ratio $\nu = 0.25$

   \item \textbf{Boundary conditions}\\
   Clamped boundary conditions at node 0, 9 and 18, i.e.\ left side of the cantilever beam.

   \item \textbf{Loading}\\
   A concentrated load of $40 kips$ in total. Node 8 and 26 has a load of $6 \frac{2}{3}$, node 17 has a load of $26 \frac{2}{3}$.
  \end{itemize}

  \paragraph{Results:} The displacements in x- and y-direction at node 22 and 26 are presented in Table \ref{tab:testB}. The results of \textit{SAP2000} were used for comparison. The displacements calculated by the thesis' program deviate from the commercial software in all cases for at most $0.03\%$ and thus can be accepted as satisfactory accurate.

  \begin{table}[htbp]
   \centering
    \begin{tabular}{c|c|C{2.5cm}|C{2.5cm}|c}
    \textbf{Node} & \textbf{Displacement} & \textbf{Results from program} & \textbf{Results from SAP-2000} & \textbf{Difference (\%)}\\\hline\hline
    \multirow{2}{*}{22} & $u_x$ & $-0.0427728$ & $-0.042774$ & $0.028\%$\\
                        & $u_y$ & $ 0.1012620$ & $ 0.101265$ & $0.030\%$\\\hline
    \multirow{2}{*}{26} & $u_x$ & $-0.0570728$ & $-0.057074$ & $0.021\%$\\
                        & $u_y$ & $ 0.3160560$ & $ 0.316064$ & $0.025\%$\\\hline
    \end{tabular}
   \caption{Displacements and deviations for Test B}
   \label{tab:testB}
   \end{table}
     
 \subsection{Test C: Plate Displacement with Tri-3}
  In this test the three-node triangular plate element \textbf{Tri-3} was validated. The example problem was taken from \cite{wilson1996three} (8.9.2). The geometry of the test is shown in Figure \ref{fig:testC}. Four different tests were made: At first the square plate were subdivided into $4\!\times\!4$ squares. Each square were divided into two triangles by a diagonal. Here, two possibilities are given: From the upper left to the lower right ($\boxbackslash$) or from the upper right to the lower left ($\boxslash$). This variant is shown in Figure \ref{fig:testC}. The other two tests were made with the same triangle variants but with a $16\!\times\!16$ mesh subdivision.
  \begin{figure}[htbp]
  	\centering
  	\setlength\unitlength{1.05cm}
  	\begin{picture}(7,7.5)
  	\thicklines
  	\put(0.5,0.5){\vector(1,0){1}}
  	\put(0.5,0.5){\vector(0,1){1}}
  	\put(1.6,0.6){$\mathbf{x}$}
  	\put(0.6,1.6){$\mathbf{y}$}   	
  	\thinlines
  	\polygon(1,1)(7,1)(7,7)(1,7)
  	\Line(1,4)(7,4)\Line(4,1)(4,7)
  	\Line(1,2.5)(7,2.5)\Line(2.5,1)(2.5,7)
  	\Line(1,5.5)(7,5.5)\Line(5.5,1)(5.5,7)
  	\Line(1,1)(7,7)\Line(1,2.5)(5.5,7)\Line(1,4)(4,7)\Line(1,5.5)(2.5,7)\Line(2.5,1)(7,5.5)\Line(4,1)(7,4)\Line(5.5,1)(7,2.5)
  	\polygon(1,1)(1.1,0.8)(0.9,0.8)
  	\polygon(1,1)(0.8,1.1)(0.8,0.9)
  	\polygon(7,1)(7.1,0.8)(6.9,0.8)
  	\polygon(7,1)(7.2,1.1)(7.2,0.9)
  	\polygon(7,7)(7.1,7.2)(6.9,7.2)
  	\polygon(7,7)(7.2,7.1)(7.2,6.9)
  	\polygon(1,7)(1.1,7.2)(0.9,7.2)
  	\polygon(1,7)(0.8,7.1)(0.8,6.9)
  	\put(4,4){\circle*{0.25}} \put(4.1,3.65){$P$}
  	\put(1.06,1.05){$0$}\put(2.56,1.05){$1$}\put(4.06,1.05){$2$}\put(5.56,1.05){$3$}\put(7.06,1.05){$4$}
  	\put(1.06,2.55){$5$}\put(2.56,2.55){$6$}\put(4.06,2.55){$7$}\put(5.56,2.55){$8$}\put(7.06,2.55){$9$}
  	\put(1.06,4.05){$10$}\put(2.56,4.05){$11$}\put(4.06,4.05){$12$}\put(5.56,4.05){$13$}\put(7.06,4.05){$14$}
  	\put(1.06,5.55){$15$}\put(2.56,5.55){$16$}\put(4.06,5.55){$17$}\put(5.56,5.55){$18$}\put(7.06,5.55){$19$}
  	\put(1.06,7.05){$20$}\put(2.56,7.05){$21$}\put(4.06,7.05){$22$}\put(5.56,7.05){$23$}\put(7.06,7.05){$24$}
  	\end{picture}
  	\caption{A square plate simply supported on all four sides is shown with a mesh subdivision of $4\!\times\!4$ squares. Each square is divided into two triangles through a diagonal going from the lower left to the upper right corner. This orientation is symbolized by ``$\boxslash$''. In the center of the plate a concentrated load $P$ is applied.}
  	\label{fig:testC}
  \end{figure}
  
  \begin{itemize}
  	\item \textbf{Mesh dimensions}\\
  	Side length $l = 10.0$\\
  	Thickness $t = 1.0$
  	
  	\item \textbf{Material properties}\\
  	Young's Modulus $E = 10.92$\\
  	Poisson's ratio $\nu = 0.3$
  	
  	\item \textbf{Boundary conditions}\\
  	All sides of the square plate are simply supported.
  	
  	\item \textbf{Loading}\\
  	A concentrated load of $1.0$ is applied on the center node of the square.
  \end{itemize}
  
  \paragraph{Results:} Wilson \cite{wilson1996three} states that the exact thin-plate displacement of the central node for this problem is $w_c = 1.16$. The results of the test is presented in Table \ref{tab:testC}. The program's result is compared to the exact value as well as to the results of the Dircrete Kirchhoff Element (DKE) presented in \cite{wilson1996three}. The DKE is also a three-node triangular plate bending element. At first, the results show that no difference exists between the different orientations of the triangles. For the $4\!\times\!4$ mesh subdivision the difference between the program's result and the benchmark is $10.69\%$ while it is only $8.69\%$ compared with the exact value. The relative big difference can be explained by the simple formulation of the Tri-3 element. For the $16\!\times\!16$ mesh subdivision the difference shrinks to only $0.72\%$ compared to the exact value and $0.97\%$ to the benchmark.  
  \begin{table}[htbp]
  	\centering
  	\begin{tabular}{C{1.5cm}|C{3.0cm}|C{2.5cm}|C{2.5cm}|C{2.7cm}}
\small\textbf{Mesh variant} & \small\textbf{Displacement at center node} & \small\textbf{Results from program} & \small\textbf{Results of DKE} & \small\textbf{Difference to 1.16 (\%)}\\\hline\hline
\multicolumn{5}{l}{$4\!\times\!4$ mesh subdivision}\\\hline
$\boxslash$     & $w_{c_{12}}$ & $1.06723$ & $1.195$ & $8.69\%$\\\hline
$\boxbackslash$ & $w_{c_{12}}$ & $1.06723$ & $1.195$ & $8.69\%$\\\hline\hline
\multicolumn{5}{l}{$16\!\times\!16$ mesh subdivision}\\\hline
$\boxslash$     & $w_{c_{144}}$ & $1.15169$ & $1.163$ & $0.72\%$\\\hline
$\boxbackslash$ & $w_{c_{144}}$ & $1.15169$ & $1.163$ & $0.72\%$\\\hline
  	\end{tabular}
  	\caption{Displacements and deviations for Test C}
  	\label{tab:testC}
  \end{table}
 \subsection{Test D: Plate Displacement with Quad-4}
 This example problem tests the four-node quadrilateral plate element \textbf{Quad-4}. The test was taken from \cite{jin1994analysis} (4.2.1, 4.2.2). Several different configurations were made: The square plate was subdivided into 4, 8 and 16 square elements on each direction. Each mesh subdivision was then tested with a concentrated load of 30000 on the central node and with a uniformly distributed load of 300 per square unit. The geometry of for one of the six test configurations is shown in Figure \ref{fig:testD}.
  \begin{figure}[htbp]
  	\centering
  	\setlength\unitlength{1.05cm}
  	\begin{picture}(8,8)
  	\thicklines
  	\put(0.5,0.5){\vector(1,0){1}}
  	\put(0.5,0.5){\vector(0,1){1}}
  	\put(1.6,0.6){$\mathbf{x}$}
  	\put(0.6,1.6){$\mathbf{y}$}   	
  	\thinlines
  	\polygon(1,1)(7,1)(7,7)(1,7)
  	\Line(1,1.75)(7,1.75)\Line(1.75,1)(1.75,7)
  	\Line(1,2.50)(7,2.50)\Line(2.50,1)(2.50,7)
  	\Line(1,3.25)(7,3.25)\Line(3.25,1)(3.25,7)
  	\Line(1,4.00)(7,4.00)\Line(4.00,1)(4.00,7)
  	\Line(1,4.75)(7,4.75)\Line(4.75,1)(4.75,7)  	
  	\Line(1,5.50)(7,5.50)\Line(5.50,1)(5.50,7)
	\Line(1,6.25)(7,6.25)\Line(6.25,1)(6.25,7)
  	\polygon(1,1)(1.1,0.8)(0.9,0.8)
  	\polygon(1,1)(0.8,1.1)(0.8,0.9)
  	\polygon(7,1)(7.1,0.8)(6.9,0.8)
  	\polygon(7,1)(7.2,1.1)(7.2,0.9)
  	\polygon(7,7)(7.1,7.2)(6.9,7.2)
  	\polygon(7,7)(7.2,7.1)(7.2,6.9)
  	\polygon(1,7)(1.1,7.2)(0.9,7.2)
  	\polygon(1,7)(0.8,7.1)(0.8,6.9)
  	\put(4,4){\circle*{0.25}} \put(4.1,3.65){$P$}
  	\put(1.06,1.05){$0$}\put(4.06,1.05){$4$}\put(7.06,1.05){$8$}
  	\put(1.06,2.55){$18$}\put(4.06,2.55){$22$}\put(7.06,2.55){$26$}
  	\put(1.06,4.05){$36$}\put(4.06,4.05){$40$}\put(7.06,4.05){$44$}
  	\put(1.06,5.55){$54$}\put(4.06,5.55){$58$}\put(7.06,5.55){$62$}
  	\put(1.06,7.05){$72$}\put(4.06,7.05){$76$}\put(7.06,7.05){$80$}
  	\end{picture}
  	\caption{The mesh of test D is shown with a $8\!\times\!8$ subdivision and concentrated central loading configuration. The mesh can be subdivided coarser of finer and the loading can be applied uniformly over the whole plate. The square plate is in all cases simply supported on all four sides.}
  	\label{fig:testD}
  \end{figure}
  
  \begin{itemize}
   \item \textbf{Mesh dimensions}\\
   Side length $l = 10.0$\\
   Thickness $t = 0.5$
    	
   \item \textbf{Material properties}\\
   Young's Modulus $E = 10^7$\\
   Poisson's ratio $\nu = 0.3$
    	
   \item \textbf{Boundary conditions}\\
   All sides of the square plate are simply supported.
    	
   \item \textbf{Loading}\\
   A uniform load of $300$ is applied on the whole plate in a first test, while a concentrated load of $30,000$ is applied on the center node of the square in a second test.
  \end{itemize}
    
  \paragraph{Results:} For the test with the uniform load, \cite{jin1994analysis} states that the displacement of the central node $w_c^*$ can be evaluated exactly with the help of the plate theory, that is:
  \begin{equation}
  w_c^* = 0.00406 \frac{q_0 a^4}{D}
  \end{equation}
  With $q_0 = 300$ being the uniform load, $a = 10$ the length of the square plate and $D = \frac{E t^3}{12 (1-\nu^2)}$ the material property, the central node's displacement follows as:
  \begin{equation}
  w_c^* = 0.00406 \frac{300 * 10^4}{\frac{10^7 0.5^3}{12(1-0.3^2)}} = 0.1064045
  \end{equation}
  The results with the different mesh subdivision levels are very accurate, the $8\!\times\!8$-subdivision if less than a thousands percents different to the analytical solution and even the coarse level with only 4 elements per direction has only $0.35\%$ difference to the exact value.\\
  For the concentrated load test, \cite{jin1994analysis} also proposes an analytic solution, namely:
  \begin{equation}
  w_c^* = 0.0115999 \frac{P a^4}{D}
  \end{equation}
  where $P = 30000$ is the concentrated load at the center node. The rest of the variables stays the same as in the uniform load test. The analytical solution to this example problem is then $
  w_c^* = 0.30401019$. The program's results are far more inaccurate compared to the uniform load test with a maximum difference of $8.62\%$. The fine subdivision of $16\!\times\!16$ elements is though acceptably accurate with less than a percent difference to the analytically exact value.
  \begin{table}[htbp]
   \centering
   \begin{tabular}{C{2.0cm}|C{3.0cm}|C{2.5cm}|C{2.5cm}|C{2.7cm}}
\small\textbf{Mesh subdivisions} & \small\textbf{Displacement at center node} & \small\textbf{Results from program} & \small\textbf{Analytical solution} & \small\textbf{Difference (\%)}\\\hline\hline
  \multicolumn{5}{l}{Uniform loading}\\\hline
  $4\!\times\!4$   & $w_{c_{12}}$ & $0.106032$ & \multirow{3}{*}{$0.1064045$} & $0.35\%$\\\cline{1-3}\cline{5-5}
  $8\!\times\!8$   & $w_{c_{40}}$ & $0.106405$ &  & $0.00047\%$\\\cline{1-3}\cline{5-5}
  $16\!\times\!16$ & $w_{c_{144}}$& $0.106454$ &  & $0.046\%$\\\hline\hline
  \multicolumn{5}{l}{Concentrated loading}\\\hline
  $4\!\times\!4$   & $w_{c_{12}}$ & $0.332677$ & \multirow{3}{*}{$0.30401019$} & $8.62\%$\\\cline{1-3}\cline{5-5}
  $8\!\times\!8$   & $w_{c_{40}}$ & $0.312851$ &  & $2.83\%$\\\cline{1-3}\cline{5-5}
  $16\!\times\!16$ & $w_{c_{144}}$& $0.306664$ &  & $0.86\%$\\\hline\hline
    	\end{tabular}
    	\caption{Displacements and deviations for Test D}
    	\label{tab:testD}
    \end{table}
 \subsection{Test E: Shell Displacement}
 %Ein H-Trägerbalken. Am einen Ende fest eingespannt. Am anderen Ende wird oben eine Kraft am äußeren Knoten in den Balken hinein in flacher Ebene gegeben, gleichzeitig wird unten an der gegenüberliegenden Seite eine Kraft in entgegengesetzter Richtung gegeben\newline
 %test\_j\_tri.xda - korrekt
 %Gleich wie Test E nur eben Quadelemente\newline
 %test\_j\_quad.xda - korrekt
 \subsection{Test F: Convergence (increasing number of elements)}
  \begin{figure}[htbp]
  	\centering
  	\setlength\unitlength{1.05cm}
  	\begin{picture}(12,4)
  	\thicklines
  	\put(0.5,0.5){\vector(1,0){1}}
  	\put(0.5,0.5){\vector(0,1){1}}
  	\put(1.6,0.6){$\mathbf{x}$}
  	\put(0.6,1.6){$\mathbf{y}$}   	
  	\thinlines
  	\polygon(1,1)(11,1)(11,3)(1,3)
  	\Line(1,1.5)(11,1.5)\Line(3.5,1)(3.5,3)
  	\Line(1,2)(11,2)\Line(6,1)(6,3)
  	\Line(1,2.5)(11,2.5)\Line(8.5,1)(8.5,3)
  	\Line(1,3)(11,1)\Line(3.5,3)(11,1.5)\Line(6,3)(11,2)\Line(8.5,3)(11,2.5)
  	\Line(1,2.5)(8.5,1)\Line(1,2)(6,1)\Line(1,1.5)(3.5,1)
  	\polygon(1,1)(1.1,0.8)(0.9,0.8)
  	\polygon(1,1)(0.8,1.1)(0.8,0.9)
  	\polygon(11,1)(11.1,0.8)(10.9,0.8)
  	\polygon(11,1)(11.2,1.1)(11.2,0.9)
  	\polygon(11,3)(11.1,3.2)(10.9,3.2)
  	\polygon(11,3)(11.2,3.1)(11.2,2.9)
  	\polygon(1,3)(1.1,3.2)(0.9,3.2)
  	\polygon(1,3)(0.8,3.1)(0.8,2.9)
  	\put(6,2){\circle*{0.25}} \put(6.1,1.65){$P$}
  	\put(1.06,1.05){$0$}\put(6.06,1.05){$2$}\put(11.06,1.05){$4$}
  	\put(1.06,2.05){$10$}\put(6.06,2.05){$12$}\put(11.06,2.05){$14$}
  	\put(1.06,3.05){$20$}\put(6.06,3.05){$22$}\put(11.06,3.05){$24$}
  	\end{picture}
  	\caption{}
  	\label{fig:testF}
  \end{figure}
 \subsection{Test G: MPI (increasing number of processes)}
 %??? theoretisch alle Tests, z.B. E/F mit Prozessoranzahl = 1,2,4,8,16. In dem Fall ist natürlich die Zeit interessant und ob die Ergebnisse jeweils alle gleich sind
 \subsection{Test H: Coupling with preCICE}
 %???
\newpage