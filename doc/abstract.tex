\section*{Abstract}
% - Minimum halbe Seite bzw. 250-300 wörter
% - Was habe ich gemacht ... is presented.
% - vergleichaspekte erstellt und frameworks getestet, um am besten geignetes zu finden -> libMesh ist es geworden
% - Detailierter, welche Shell Elements, bestehen aus Plane und Plate
% - 2 Versionen: Tri und Quad, jeweils welches Modell
% - Programm entwickelt: eigenständige version und version mit einbindung von precice zur kopplung an multi-physics simulation. das programm kann mittels mpi parallel betrieben werden
% - details zur implementierung 
% - validierung der shell elements und ihrer komponenten durch viele tests. tests haben hohe genauigkeit in plane displacements gezeigt und ausreichend gute genauigkeit in plate displacements. die genauigkeit der shell elements ist akzeptabel aufgrund der einfachen approximation der elemente. die konvergenz-tests zeigen, dass bei ausreichend hoher unterteilung des meshs die genauigkeit sehr hoch ist.
% - das programm skaliert gut im parallelen
% - die kopplungstests/die validierung der kopplung hat gezeigt, dass das programm mittels precice erfolgreich an multi-physic simulationen gekoppelt werden kann

The development of a FEM structure solver for a coupled fluid-structure interaction simulation is presented in this thesis. Different aspects were created to evaluate existing FEM libraries. The libMesh framework was considered to be best suitable and was used in the development of the program. Two different discretizations of flat shell elements were implemented, a triangular and a quadrilateral element. The shell elements are constructed by the superposition of plane and plate elements. For the plane elements a model from XXX and XXX was used, the plate element's model is XXX and XXX. The developed program offers two versions, one coupled version to be used in a multi-physics simulation and a stand-alone version whose intention was to better validate the implemented finite elements. Both versions are parallelized with MPI. The coupling environment preCICE was used in the development of the coupled version. The validation of the elements showed good accuracy for the plane element components compared to analytical solutions as well as commercially available software. The plate element's accuracy is lower compared to the plane elements due to the chosen models that have a simpler approximation of the physical circumstances. The superimposed shell element's accuracy is well suited to be used in the structure solver. For both finite elements the accuracy can be increased arbitrarily by subdividing the mesh further. The parallelization test showed a good scaling with the number of processes for the assembly of the system's matrix and right-hand side as well as for the solving step. As an coupling example, a fluid-structure coupling between the developed program and a fluid solver developed with openFOAM was tested. The developed structure solver showed good performance in the simulation and is qualified for further multi-physics simulations connected via preCICE.
\newpage