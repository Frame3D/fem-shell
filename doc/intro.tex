\section{Introduction}
The finite element method (FEM) is a method for finding numerical approximate solutions to field problems for partial differential equations~\cite{cook2002concepts}. It partitions the whole problem domain into small parts, called finite elements, distinguishing them from infinitesimal elements used in calculus. Since the spatial variation of each element is described in simpler form, for example by polynomial terms, FEM only provides an approximate solution. Despite the fact of inexact solutions by FEM, it is applicable to many fields in practice: Fluid dynamics, heat transfer, stress analysis or magnetic fields, for instance. The analyzed structure is not limited to any shape and by refining the mesh, the approximation can be improved easily. Therefore, FEM is widely used in fluid and structure simulations~\cite{cook2002concepts}.

While simulations of fluids and structures for their own are present for many decades, fluid-structure interactions (FSI) are much younger due to the computational complexity when combining both physical fields in one simulation~\cite{gatzhammer2015efficient}. Nowadays, FSI simulations are not the only examples of multi-physics simulations: Fluid-structure-acoustics (\cite{link20092d},~\cite{schafer2010fluid}) even combine three physical fields in one simulation. When dealing with multi-physics problems one has two options: One can hold all parts of the problem together in one application code, independent from other computing applications. This is also called the monolithic approach~\cite{gatzhammer2015efficient}. The other option is to break the problem apart into smaller pieces that can be better managed in development and recombine them to a coupled solution. This is denoted as loose coupling or partitioned approach~\cite{lohner2006extending}. The challenge of the second option is to preserve stability of the simulation. The benefits on the other hand, are the possibilities to reuse existing code and to encapsulate the solver codes from each other and the coupling functionality. With preCICE, a coupling tool was introduced by~\cite{gatzhammer2015efficient} that eases the combination of multiple different solver codes into one simulation, following the idea of the partitioned approach. It is used in this thesis to realize the coupling with other fluid solvers.

% - ziele und durchführung (ein bisschen den abstract wiederholen, aber keine ergebnisse mehr verraten)
The goal of this work is to develop a structure solver with FEM that is applicable to a multi-physics simulations like a fluid-structure coupling. The program uses a FEM library, which provides data structures and functionalities to support programming parts like the initialization, setup and solving of the system. In order to find a suitable framework, an evaluation based on many aspects is performed on different FEM frameworks in the thesis.

The solver works on two-dimensional meshes consisting of triangular and quadrilateral elements. Within the scope of this thesis flat shell elements are implemented. A flat shell element can be constructed by superimposing a plane and a plate element. For this reason, six finite element models are implemented: One plane, plate and shell element sharing a three-node triangular finite element approach, the others sharing a four-node quadrilateral finite element approach. The triangular and quadrilateral plane element is implemented based on the model description from~\cite{steinke2005finite}. The triangular plate element is based on a model from~\cite{specht1988modified}, while the quadrilateral plate elements implements the Discrete Kirchhoff Quadrilateral (DKQ) element introduced by~\cite{zienkiewicz2000finite}. Both plate elements belong to the group of models sharing the thin-plate theory of Kirchhoff~\cite{steinke2005finite}. The shell elements are then constructed by superimposing the corresponding plane and plate elements and the adding a sixth degree of freedom for the twist in z-direction, that cannot be modeled with the plane and plate elements implemented in this thesis.

The implementation of the FEM-code gets illustrated in detail, including practical examples of features of the FEM framework that was chosen in the evaluation. The program is parallelized with the help of MPI, while the preCICE API is integrated to make it applicable for multi-physics coupling simulations. A validation of the implemented models as well as the parallelization demonstrate the correctness and accuracy of the implementation and benefits from parallel execution. The coupling through preCICE is tested in two additional tests. One of the tests is a fluid-structure interaction, where a beam is put into the cross-flow of a channel. For this, the thesis' program is coupled with an external fluid solver.

\subsection{Organization}
% was gibt es in welchem kapitel für inhalt, quasi ein überblick über die arbeit ohne genaue details oder ergebnisse
The thesis is made up of seven chapters including the introduction. A detailed list of aspects used to evaluate several FEM frameworks is presented in \textbf{Chapter~2}, along with short outlines of the surveyed libraries and a motivation of the chosen library. \textbf{Chapter~3} contains the mathematical derivations of the shell elements. After a motivation of linear elastic problems, first the plane element in its triangular and quadrilateral form is described, followed by the plate elements. Before the construction of the shell elements, a section about coordinate transformation is presented. \textbf{Chapter~4} describes details on the implementation process of the FEM code. First, an introduction to libMesh is shown. The remainder of this chapter describes the single parts of the code, like the mesh file import or the system matrix assembly, as well as the parallelization of the program with MPI. In \textbf{Chapter~5} the coupling via preCICE is presented. An overview of preCICE is given and details on the coupling with it is described, including the different coupling methods, data mappings and communication methods it provides. The integration of preCICE into the existing structure solver code closes this chapter. \textbf{Chapter~6} contains many example problems with the help of which the shell elements and its components get validated. Also, the parallelization and the coupling is tested in this chapter. \textbf{Chapter~7} presents a summary and a discussion of the results and gives suggestions for future development.