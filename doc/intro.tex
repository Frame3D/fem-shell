\section{Introduction}
% Batoz_et_al-1982 \cite{batoz1982evaluation}

% - einführende worte zu FEM -> see \cite{kansara2004development}
%
% - ziele und durchführung (ein bisschen den abstract wiederholen, aber keine ergebnisse mehr verraten)
The goal of this work is to develop a structure solver with the finite element method that is part of a multi-physics simulations like a fluid-structure interaction coupling. The program uses a FEM library that provides data structures and functionality to ease the initialization, setup and solving of the system. Therefore, an evaluation based on several aspects is performed on different FEM frameworks.

The solver works on two-dimensional meshes consisting of triangular and quadrilateral elements. In the scope of this thesis flat shell elements are implemented. A flat shell element can be constructed by superimposing a plane and a plate element. For this reason six finite element models is implemented: One plane, plate and shell element sharing a three-node triangular finite element approach, the others sharing a four-node quadrilateral finite element approach. The triangular and quadrilateral plane element is implemented based on the model description from \cite{steinke2005finite}. The triangular plate element is based on a model from \cite{specht1988modified}, while the quadrilateral plate elements implements the Discrete Kirchhoff Quadrilateral (DKQ) described in \cite{zienkiewicz2000finite}.

\subsection{Organization}
% was gibt es in welchem kapitel für inhalt, quasi ein überblick über die arbeit ohne genaue details oder ergebnisse

\newline